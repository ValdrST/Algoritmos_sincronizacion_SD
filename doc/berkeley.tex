En muchos algoritmos como el de NTP, el servidor de tiempo es pasivo. Otras
máquinas solicitan periódicamente el tiempo. Todo lo que se hace es responder a
sus consultas. En Berkeley UNIX se aplica exactamente el método opuesto.
Aquí, el servidor de tiempo (de hecho, un demonio de tiempo) es activo, ya que
cada cierto tiempo pregunta a cada máquina sobre la hora ahí registrada
(\cite{TSO}).

El algoritmo de Berkeley es una técnica de sincronización de reloj utilizada en
sistemas distribuidos. El algoritmo asume que cada nodo de la máquina en la red
no tiene una fuente de tiempo precisa o no posee un servidor UTC.\@

En la \textbf{Clase Master} se elige un nodo individual como nodo maestro de un
grupo de nodos en la red. Este nodo es el nodo principal de la red que actúa
como maestro y el resto de los nodos actúan como esclavos. El nodo maestro se
elige mediante un proceso de elección/algoritmo de elección de líder.
También se hace una subrutina utilizada para la diferencia de reloj
promedio
\begin{listing}[H]
\inputminted[
  frame=lines,
  framesep=2mm,
  baselinestretch=1.2,
  bgcolor=LightGray,
  fontsize=\scriptsize,
  linenos,
  firstline=36%, lastline=51
  ]{python}{../berkeley/Master.py}
\caption{Master haciendo ajustes a los Slaves}
\label{lst:1}
\end{listing}

En la \textbf{Clase Slave} los slaves se inicializan y se les asigna el tiempo
en que fueron inicializados, ese tiempo se los asigna la clase master utilizando
la biblioteca \mintinline{shell}{zmq}.
Se utiliza \textbf{localhost}\footnote{Nombre predeterminado que se usa para establecer
  una conexión con la computadora local computadora usando la red de dirección
  de bucle invertido. $127.0.0.1$} y los puertos $1000$, $1001$, $1002$ y $1004$.
\begin{listing}[H]
\inputminted[
  frame=lines,
  framesep=2mm,
  baselinestretch=1.2,
  bgcolor=LightGray,
  fontsize=\scriptsize,
  linenos,
  firstline=11,
  %lastline=49
  ]{python}{../berkeley/Slave.py}
\caption{Clase Slave}
\label{lst:2}
\end{listing}

\begin{figure}[H]
  \centering
  \includegraphics[width=10cm]{1}
  \caption{Solicitud de tiempo para cada Slave y el mismo Master}\label{fig:1}
\end{figure}

Después de la sincronización, cada proceso imprime su reloj lógico para
verificar el resultado de la sincronización.

El Slave maestro calcula la diferencia de tiempo promedio entre todas las horas
de reloj recibidas y la hora de reloj proporcionada por el reloj del sistema
maestro. Esta diferencia de tiempo promedio se agrega a la hora actual en el
reloj del sistema principal y se transmite a través de la red.

\begin{figure}[H]
  \centering
  \includegraphics[width=10cm]{2}
  \caption{Sincronización del tiempo}\label{fig:2}
\end{figure}
\begin{listing}[H]
\inputminted[
  frame=lines,
  framesep=2mm,
  baselinestretch=1.2,
  bgcolor=LightGray,
  fontsize=\scriptsize,
  linenos,
  firstline=7,
  %lastline=49
  ]{python}{../berkeley/Workbench.py}
\caption{Implementación}
\label{lst:2}
\end{listing}

\begin{figure}[H]
  \centering
  \includegraphics[width=8cm]{3}
  \caption{Ejecución de \mintinline{shell}{Workbrench.py}}\label{fig:3}
\end{figure}
